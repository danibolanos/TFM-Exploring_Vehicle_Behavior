% RESUMEN EN ESPAÑOL
%%%%%%%%%%%%%% RESUMEN PROPUESTO -Mas acorde con el máster %%%%%%%%%%%%%%%%%%%%%%%%%%%%%

Actualmente, debido al cambio climático y al incremento de turistas en todos los entornos, los gestores de organizaciones ubicadas en parajes vulnerables, y los políticos deben aplicar técnicas de gestión de recursos humanos, materiales y políticas de protección medioambiental, al igual que aplican las empresas en sus gestiones rutinarias, cumpliendo con las leyes medioambientales y siendo proactivos en la aplicación de otras técnicas no obligatorias. 

El primer paso para aplicar estas políticas pasa por conocer la situación actual. En este sentido, este TFM aborda el conocimiento del tráfico que circula en un paraje vulnerable, como son los pueblos del valle de Poqueira, situados a los pies del paraje natural de Sierra Nevada. En concreto, hemos realizado un proyecto en el cual se ha diseñado el despliegue de sensores de tráfico, gestionando la implementación con la empresa de sensores y definiendo estrategias para la extracción de conocimiento a partir de los datos recopilados. Posteriormente, se aplican técnicas de aprendizaje automático sobre los datos recogidos para estudiar los patrones de tráfico.

% Al mejorar el conjunto de datos con valores calculados, extraemos información valiosa sobre diversos patrones de movilidad. 

Proponemos un pipeline de análisis de patrones que utiliza múltiples fuentes de datos y simplifica la selección de algoritmos de agrupación y normalización. En nuestra investigación, destacamos la importancia de elegir el algoritmo de normalización adecuado para escalar eficientemente las características de entrada de datos heterogéneos. Esto es fundamental para comprender y analizar los patrones de movilidad. Para validar nuestro enfoque, utilizamos datos de cuatro cámaras de reconocimiento de matrículas (LPR) recopilados durante nueve meses en una comarca de la Alpujarra Granadina. También incorporamos bases de datos adicionales con información sobre origen, ingresos y datos de vacaciones, lo que nos dio un conjunto de datos de más de 50.000 vehículos.

Aplicando nuestro pipeline y analizando este gran conjunto de datos, identificamos diversos patrones de tráfico entre residentes y visitantes en una zona turística rural. Los resultados de nuestro estudio aportan valiosas ideas a los analistas de datos sobre los factores a tener en cuenta a la hora de seleccionar algoritmos adecuados para analizar conjuntos de datos heterogéneos. Nuestros resultados son de suma importancia tanto para los gestores de parques nacionales, que se encuentran en zonas especialmente vulnerables a las amenazas derivadas del cambio climático, como para los gestores de organizaciones ubicadas en estos parajes. Además, proporcionan a los responsables políticos una comprensión más profunda de los patrones de movilidad en áreas sensibles desde el punto de vista medioambiental. Todo ello facilita la consecución de la sostenibilidad de estos territorios en una triple vertiente: económica, social y medioambiental.

%%%%%%%%%%%%%% FIN RESUMEN PROPUESTO -Mas acorde con el máster %%%%%%%%%%%%%%%%%%%%%%%%%%%%%



%%%%%%%%%%%%%% RESUMEN DANIEL %%%%%%%%%%%%%%%%%%%%%%%%%%%%%
%Este trabajo de fin de máster explora el análisis de los patrones de movilidad en los pueblos inteligentes, con un enfoque específico en el comportamiento del coche. Mientras que los estudios anteriores se han centrado principalmente en las ciudades inteligentes, hay una falta de análisis en el contexto de los pueblos inteligentes y el uso de algoritmos óptimos para el análisis de datos. Para llenar este vacío, proponemos un pipeline que integra múltiples fuentes de datos y facilita la selección de algoritmos de agrupación y normalización. Al mejorar el conjunto de datos con valores calculados, extraemos información valiosa sobre diversos patrones de movilidad.

%Nuestra investigación pone de relieve la importancia de seleccionar el algoritmo de normalización adecuado para escalar eficientemente las características de entrada a partir de fuentes de datos heterogéneas. Este aspecto desempeña un papel clave en la comprensión y el análisis de los patrones de movilidad. Para validar nuestro enfoque, utilizamos datos de cuatro cámaras de reconocimiento de matrículas (LPR), recogidos durante un periodo de nueve meses en una comarca de la Alpujarra Granadina. Además, incorporamos varias bases de datos que contienen información como procedencia, ingresos brutos y datos de vacaciones, dando como resultado un conjunto de datos de más de 50.000 vehículos.

%Aplicando nuestro pipeline y analizando este gran conjunto de datos, identificamos diversos patrones de tráfico entre residentes y visitantes en una zona turística rural. Los resultados de nuestro estudio aportan valiosas ideas a los analistas de datos sobre los factores a tener en cuenta a la hora de seleccionar algoritmos adecuados para analizar conjuntos de datos heterogéneos. Nuestros resultados son de suma importancia tanto para los gestores de parques nacionales, que se encuentran en zonas especialmente vulnerables a las amenazas derivadas del cambio climático, como para los gestores de organizaciones ubicadas en estos parajes. Además, proporcionan a los responsables políticos una comprensión más profunda de los patrones de movilidad en áreas sensibles desde el punto de vista medioambiental. Todo ello facilita la consecución de la sostenibilidad de estos territorios en una triple vertiente: económica, social y medioambiental.
%%%%%%%%%%%%%% FIN RESUMEN DANIEL %%%%%%%%%%%%%%%%%%%%%%%%%%%%%