% RESUMEN EN INGLÉS

Currently, due to climate change and the increase of tourists in all environments, managers of organizations located in vulnerable places and politicians must apply human resource management techniques, materials and environmental protection policies, just as companies do in their routine management, complying with environmental laws and being proactive in the application of other non-mandatory techniques. 

The first step to apply these policies is to know the current situation. In this sense, this master's thesis deals with the knowledge of the traffic that circulates in a vulnerable place, such as the villages of the Poqueira Valley, located at the foot of the natural site of Sierra Nevada. Specifically, we have carried out a project in which we have designed the deployment of traffic sensors, managed the implementation with the sensor company and defined strategies for the extraction of knowledge from the collected data. Subsequently, machine learning techniques are applied on the collected data to study traffic patterns.

We propose a pattern analysis pipeline that utilizes multiple data sources and simplifies the selection of clustering and normalization algorithms. In our research, we highlight the importance of choosing the right normalization algorithm to scale the input features of heterogeneous data efficiently. This is critical for understanding and analyzing mobility patterns. To validate our approach, we use data from four license plate recognition (LPR) cameras collected over nine months in a district of the Alpujarra Granadina. We also incorporated additional databases with information on origin, income, and vacation data, giving us a dataset of more than 50,000 vehicles.

By applying our pipeline and analyzing this large dataset, we identified diverse traffic patterns between residents and visitors in a rural tourist area. The results of our study provide valuable insights to data analysts on factors to consider when selecting suitable algorithms for analyzing heterogeneous datasets. Our findings are of utmost importance for both managers of national parks, which are particularly vulnerable to climate change-related threats, and managers of organizations located in these areas. Furthermore, they provide policymakers with a deeper understanding of mobility patterns in environmentally sensitive areas. This, in turn, facilitates the achievement of sustainability in these territories across three dimensions: economic, social, and environmental.