\chapter{Conclusions} \label{ch:future}

In this study, we have proposed a pipeline to merge data from different sources in smart villages and analyzed these data to infer the traffic behavior in the area, providing policymakers the first step to understanding their traffic, which could lead to implementing effective policies aim towards a sustainable tourism. By doing so, we have accomplished the objectives outlined in Section \ref{objetivos} regarding the investigation and analysis of traffic behavior in smart villages. The following objectives were successfully attained:

\begin{itemize}
    \item We analyzed the suitability of specific algorithms for different data distributions, enabling us to make informed decisions during the analysis process (see in Chapter \ref{ch:fundamentals}).
    \item We extensively reviewed previous works on clustering applied to mobility patterns, information fusion, and traffic management, helping us leverage existing knowledge and identify potential gaps in the current research landscape (see in Chapter \ref{ch:related_work}).
    \item We discussed the design and deployment process of the infrastructure used to collect data on vehicle movements in smart villages (see in Section \ref{sec:setup}).
    \item We analyzed the collected data and performed statistical analyses to identify patterns and correlations among variables from different data sources (see in Chapter \ref{ch:results}).
    \item We evaluated various normalization techniques to preprocess the collected data, considering the effectiveness of each technique in improving result quality (see in Chapter \ref{ch:results}).
    \item We applied the studied clustering algorithms to the collected data and evaluated the visualization and explainability of the results, aiming to select the most suitable algorithms for our analysis (see in Chapter \ref{ch:results}).
    \item We assessed the performance of the proposed clustering pipeline and the selected normalization methods. We discussed the different segmentations obtained and the usability of each studied method (see in Chapter \ref{ch:discussions}).
\end{itemize}

By achieving these objectives, our work contributes to understanding traffic behavior in smart villages. The information gained from this research can help improve traffic flow, enhance visitor experience, and optimize resource allocation in these regions. Our findings serve as a valuable resource for the design and implementation of effective traffic and tourism management systems and policies in smart villages.

\section{Limitations}

One of the main limitations of this study lies in the inability to validate the unregistered residents identified through the segmentation of the problem. These unregistered residents are identified through mobility patterns extracted from available data sources. However, due to the lack of real information about them, their participation in such patterns cannot be verified. The lack of labels or real information about the data points is a common limitation in unsupervised learning algorithms. These algorithms are based on finding patterns and structures in the data without the guidance of predefined labels. However, when applying these methodologies in practical situations, it is important to recognize this limitation and consider it when interpreting the results and the practical implications of the study. There is a need for further research and ways to obtain more complete data, e.g., through questionnaires, that would allow an accurate understanding of mobility patterns in areas where the unregistered population may play a significant role.

Another important limitation of this study is the use of only three principal components (PCA) to represent the data in the analysis of mobility patterns. Cluster visualization is a crucial aspect in the analysis performed, but it is clear that we cannot visualize more than three dimensions using conventional plots. When considering Andrew Curves as an alternative to visualize more than three components, we face another limitation. These curves allow us to represent multiple variables and their relationships by plotting each observation as a curve on a two-dimensional graph. However, in problems with a large amount of data, it is difficult to visually interpret Andrew curves due to the overlapping and increasing visual complexity as more variables are added. This makes it difficult to identify and understand more complex and subtle mobility patterns that might be present in the data. Given this limitation, it is important to recognize that the representation and visualization of mobility patterns in this study are restricted by the reduced dimensionality and visualization techniques used. For a more detailed understanding of mobility patterns, it would be advisable to explore more advanced dimensionality reduction and visualization techniques that can address these challenges and provide a more complete and accurate representation of mobility patterns in future research.

\section{Future Work}

As future work, the project itself has offered us interesting opportunities to enrich and expand the research conducted in this work. One possible direction is to cross-reference the data used in this study with information generated by additional sensors (waste and motion sensors to detect persons in local businesses and streets), which have also been implemented in the project, and have the potential to provide valuable information that would complement existing vehicle data.

On the one hand, the incorporation of waste sensors would allow the analysis of waste generation and management patterns in relation to the identified mobility patterns. By correlating the amount and type of waste generated with traffic activity in a specific area, deeper insights into the environmental impact and sustainable waste management needs in the context of urban mobility could be gained. This cross-referenced information could be useful to develop more efficient collection and recycling strategies.

On the other hand, the use of motion sensors that collect information on incoming and outgoing people counts and gauging could provide additional information on the mobility patterns of residents and visitors. By analyzing the number of people present at certain locations at different times, a more detailed understanding of people flows and their relationship to vehicle traffic could be obtained. This would allow the identification of time intervals, where pedestrian and vehicular mobility patterns are intertwined. These findings would be valuable for planning and designing urban infrastructures that promote sustainable mobility and transportation efficiency, taking into account both pedestrians and drivers.

Future work in this field could gain significant benefits from integrating data from waste and motion sensors. This integration would allow a more complete understanding of mobility patterns in smart villages by providing relevant information on environmental aspects, waste management and human behavior. It would be desirable to explore techniques and methodologies to combine and analyze these additional data in order to obtain a comprehensive view of urban mobility and its strategic implications. In addition, an integration of these indicators into a strategic dashboard could be carried out in order to merge the information and facilitate decision-making by local managers.