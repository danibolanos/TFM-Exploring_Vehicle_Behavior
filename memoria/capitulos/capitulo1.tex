\chapter{Introduction}

Currently, there are 13.4 billion Internet of Things (IoT) devices. Statista predicts that this figure will increase to 29.4 billion by 2030\footnote{\url{https://www.statista.com/statistics/1183457/iot-connected-devices-worldwide/}}. These devices form an interconnected network that produces extensive data in numerous social domains. Access to a large volume of data collected by various sensors makes it possible to supervise and manage different aspects of society, including healthcare, evacuation systems, smart environments, and transportation.~\cite{atzori2010internet,bermudez2018analysing,garcia2022machine, centelles2019lora}. Extracting and combining information from multiple sources, not only sensor data, but also information stored on the Internet, can lead to a better understanding of the problem to be solved, such as healthcare or vehicle mobility. For example, traffic in cities is partially dependent on local holidays. These multi-source data have resulted in the growth of some research fields, such as information fusion, intelligent environments, and ubiquitous computing. The insights from analyzing these multisource datasets can be applied to real-world problems such as tourism management, economics, and financial information systems~\cite{haughton2004sustainable}.

The number of studies with smart city data has grown exponentially in recent years. The most important cities have deployed sensor networks and IoT platforms. The data obtained by these sensors have led to numerous studies in several areas, such as traffic behavior \cite{mondal2019identifying,lin2019application,peixoto2021traffic,ning2019vehicular}. However, most solutions that try to cluster different traffic behavior do not have additional information, such as the residence of vehicle owners, to provide additional insight into the explainability of the clusters. Furthermore, this smart city trend has yet to reach small villages, and the solutions found for large cities do not always apply directly to small villages. For example, solutions monitoring traffic behavior in large cities with numerous streets and several traffic lines in some avenues do not extrapolate to villages with 6 or 7 mostly pedestrian streets and only one road with one line in each direction. Additionally, even if we try to add some explanation to the behavioral cluster in smart villages, the residency of vehicle owners is not straightforward. Due to the recent movement on moving from cities to villages and retrying or spending long periods on second residences, the actual residence information is fuzzy in rural villages.

This work proposes a clustering based on vehicle behavior in small villages, with information from license plate recognition (LPR) devices and owners' residences, among others. We applied the study directly to each individual (vehicles) and defined their spatio-temporal behavior based on their spatial frequencies of visitation. To that end, we fused several datasets and calculated new valuable variables such as the time spent in the area; total distance traveled there, etc. We study the popularly used clustering algorithms to draw conclusions on which of them performs better on the problem under consideration. In particular, we used a pipeline to analyze the particularities of the data through several visualization tools, and we explained, based on the data, the optimal normalization and clustering algorithm that best groups the different behaviors of the vehicles. We will focus on the importance of selecting the optimal normalization algorithm and its influence on the results. Additionally, we analyzed the results with residential information and determined the variables that most influence each cluster. With this information, we explained the behavior pattern of each cluster. Our pipeline comprises eight steps: data collection, cleaning, fusion, normalization, dimensionality reduction, clustering, evaluation, and visualization. Finally, we applied the proposed pipeline to a touristic rural region, with the problems mentioned above of a single small road and the lack of reliable residency information.

The findings of our research hold significant implications for policymakers, particularly in managing tourism flows in smart cities and villages. By integrating multiple data sources and employing appropriate clustering and normalization algorithms, our proposed pipeline enables a comprehensive analysis of mobility patterns. The enriched dataset, with its valuable insights into different patterns, can inform policymakers about the dynamics of tourism flows in specific areas. This knowledge is crucial for managing and optimizing transportation systems, infrastructure development, and resource allocation to accommodate the needs of both residents and visitors. By understanding the traffic patterns among residents and tourists in a rural touristic area, policymakers can devise targeted strategies to enhance visitor experiences, reduce congestion, and mitigate the environmental impact associated with tourism activities. The data analysts and policymakers can utilize our research to select suitable algorithms and gain a deeper understanding of mobility patterns, thus facilitating evidence-based decision-making and the effective management of tourism flows in environmentally sensitive areas.

\section{Motivation}

In conducting a literature review, I observed a significant gap in the analysis of smart villages and car behavior within these environments, compared to the abundance of articles focused on smart cities. In addition, there was limited exploration of normalization techniques in this context. This realization sparked a personal interest and desire to tackle the challenge of working with sensors and creating a comprehensive database from a simple one constructed from timestamps and license plate data.

Given my background, I was able to combine different fields of expertise, combining my previous knowledge in computer science with those acquired during my master's degree, such as business intelligence, data visualization and information extraction for policymakers. During my degree, I did a simple internship on clustering analysis, and dealing with the topic of unsupervised learning in the master's degree, aroused even more my interest to go deeper into this topic.

As an additional motivation, the completion of this Master's Thesis (MT) is also linked to Smart Poqueira\footnote{\url{https://wpd.ugr.es/~smartpoqueira/}} a subproject included in the project "Thematic Center on Mountain Ecosystem \& Remote sensing, Deep learning-AI e-Services" (LifeWatch-2019-10-UGR-01), linked to the analysis of different aspects related to the conservation of the Sierra Nevada National Park through advanced digital systems. The project has been co-funded by the Ministry of Science and Innovation through the ERDF funds of the Pluriregional Operational Program of Spain 2014-2020 (POPE), LifeWatch-ERIC action line, with the co-funding of the Provincial Council of Granada and the University of Granada. The heads of the Provincial Council and the University of Granada have pointed out that this commitment to the use of digital technology and innovation will make it possible to efficiently manage numerous data on the behavior of its visitors and, thus, implement solutions that improve the quality and sustainability of the visits received.

In addition to the connection with the Smart Poqueira Project, this MT is closely related to several subjects learned in the Master's program, providing a solid foundation and relevant knowledge for its development. The following are some of the areas from the master's degree courses that have a strong relationship with this dissertation:

\begin{itemize}
\item \textbf{Analysis and Inference in Business Processes}: This course has provided me with a background in machine learning techniques and validation metrics. By leveraging this knowledge, I have been able to develop intelligent systems that extract relevant information and patterns for decision-making in the field of sustainable tourism. 

\item \textbf{Business Intelligence}: Data analysis related to business strategy is a key element in implementing sustainable tourism solutions. In this subject, I have improved my skills in collecting, analyzing, and visualizing data, enabling me to make well-informed decisions for tourism management in the region of Alpujarra. 

\item \textbf{Data Bases for Business Processes}: Effective management of the data collected is crucial to the success of the project. By studying this subject, I have acquired the knowledge necessary to design and ensure the integrity and usefulness of the information collected.

\item \textbf{Big Data and Sustainability}: My understanding of big data and its implications for sustainability in the tourism industry was fostered through the subject of \textbf{Introduction to Management and Technologies in Business Processes}. Efficient management of collected data, its analysis, and the generation of insights from it will be key elements in making informed decisions and developing strategies that foster sustainability in the tourism sector.

\item \textbf{Project Management and Planing}: Effective communication and planning are essential for successful project execution. In this course, I have learned how to establish clear lines of communication with suppliers and stakeholders. In addition, this subject has given me project management techniques that support the design of project requirements and timelines essential in project development.

\item \textbf{Business Strategy and Internationalization in Technologically Advanced Environments}: In this course, I have learned that, today, organizations and businesses face a rapidly changing and highly competitive global environment. Policymakers and business owners must be agile, flexible and be able to obtain the necessary information for strategic decision-making, which facilitates higher levels of current and future competitiveness. The integration of technologies into business processes facilitates responsiveness, cost reduction, improved productivity and enhanced services.
\end{itemize}

These subjects provide a solid foundation of knowledge and skills that will be directly applied in the development of the MT. Additionally, other competencies from subjects in the program have been developed through teamwork, which has made this work possible. \textbf{Collaborative Systems and Workflow Management} (communication tools with teams and suppliers), \textbf{Dashboards and Multidimensional Systems} (designing KPIs and dashboards linked to the project), and \textbf{Design and Access to Web-Based Information and Content Management} (development of the project's website) have all contributed to the completion of this work. The combination of the connection with the Smart Poqueira project and the relevance to key subjects in the Master's program ensures the significance and applicability of this work, making a significant contribution to the advancement and development of sustainable tourism in Alpujarra.
 
\section{Objectives} \label{objetivos}

% Objetivos se han cumplido en las conclusiones

The objective of this MT is to investigate and analyze traffic behavior in smart villages using license plate recognition (LPR) devices and considering the residence of the owners. The specific objectives are as follows:

\begin{itemize}

\item Study clustering algorithms, machine learning pipelines and normalization algorithms; and the suitability of using specific algorithms for specific data distributions.

\item Review previous works on clustering applied to traffic behavior with LPR.

\item Design and deploy an infrastructure to collect data on vehicle movements in smart villages.

\item Examine, in our use case, the correlation between vehicle behavior and provenance, visitation frequency, length of stay, and seasonality.

\item Evaluate and compare different normalization techniques for preprocessing the collected data.

\item Explore various clustering algorithms and techniques and select the most appropriate ones based on the visualization and explainability of the results.

\item Evaluate the performance of the proposed clustering pipeline and selected clustering and normalization algorithms.
\end{itemize}

By achieving these objectives, this work aims to contribute to the understanding of traffic behavior in smart villages and provide valuable information for the design of effective traffic and tourism management systems and policies in the area.

\section{Thesis Outline}

% Añadir las referencias a las secciones

This chapter has presented the challenges that face the study of traffic in smart villages and the motivation of this dissertation. In Chapter \ref{ch:related_work} related work is summarized. Chapter \ref{ch:fundamentals} presents the theoretical bases discussed throughout the MT, describing the main normalization and clustering algorithms and metrics, as well as the demographic context of the area where the study was conducted. Chapter \ref{ch:methodology} presents the unsupervised learning pipeline, including the sensor's setup and the different sources of information used to construct the dataset. Chapters \ref{ch:results} show the analysis of the results. Finally, Chapter \ref{ch:discussions} concludes the work and Chapter \ref{ch:future} presents the limitations and future work related to the project.