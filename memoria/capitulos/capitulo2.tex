\chapter{Related Work} \label{ch:related_work}

\section{Information Fusion in Smart Cities for Touristic applications}

The concept of information fusion has been applied to the specific problem of tourism flows and smart cities. These approaches use advanced data analysis techniques to combine multiple sources of information, providing valuable insights for developing smart tourism applications in cities and designing sustainable environments. Smart city applications are built on top of data, and data fusion has provided a wide variety of techniques to improve the input data for an application~\cite{lau2019survey}. Examples of these techniques include data association, state estimation, unsupervised machine learning, or statistical inference. For example, combining different tourist information has been used to predict the national tourist flow in Spain with graph neural networks~\cite{saenz2023nation}. The data used in the solution are composed of tourist infrastructure information, such as camping and tourist housing from the data sources OpenStreetMap and the National Institute of Statistics of Spain (INE); reports released by the Spanish Ministry of Transportation (SGTM); and human mobility data including the number of movements between administrative areas per hour extracted from geotagged Twitter data. Most of these applications are focused either on user recommendations or tourist flow, but little attention has been paid to studying the individual behavior of the tourist inside an area (for a detailed survey, see \cite{doborjeh2022artificial, lau2019survey}). 

\section{Clustering and mobility patterns}

The increasing deployment of IoT platforms in smart cities has boosted the proliferation of sensors, including those that monitor traffic. These sensory data allow us to analyze vehicle behavior. The most common works in this area are to analyze mobility patterns in order to improve traffic congestion~\cite{mondal2019identifying,peixoto2021traffic}, and to aggregate vehicles to obtain useful conclusions for urban management~\cite{bolanos2022clustering, lin2019application}. 

To infer mobility patterns from raw data, unsupervised ML is widely adopted. In particular, various industries use clustering algorithms to categorize data into distinct groups based on similarities, differences, and patterns without prior knowledge. Clustering analysis is used to detect behavioral patterns in the field of pedestrian-vehicle mobility, and in the field of indoor-outdoor (IO) positioning systems \cite{mallik2023paving}. Some works use partitional clustering to analyze data. For example, in \cite{yao2021understanding}, the ISODATA clustering algorithm is used to cluster mobility patterns and a decision tree is used to create decision rules between the attributes and the labeling obtained from the clustering. In \cite{sun2021identifying}, they propose a K-Means clustering framework combined with other processes such as dimensionality reduction and feature extraction to classify tourists and locals based on the data generated by each individual's mobile phone signaling data and their movement through the area. Hierarchical clustering \cite{pasupathi2021trend} has also been used to segment time series related to vehicle mobility, with the objective of predicting areas where there is a higher risk of accidents. Other commonly used clustering algorithms are density-based, as they can be adapted to problems where irregular behavior occurs within the population. In \cite{belhadi2021deep}, the authors propose an alternative version of the DBSCAN clustering algorithm to detect collective anomalous human behavior from large amounts of pedestrian data in smart cities. The algorithm uses an iterative search process and aggregation of achievable density data points to form clusters, culminating in a global approach to identify behaviors of particular interest in the population. Density clustering techniques also allow the analysis of movement in areas where some specific transport behaviors are known but where more information about particular groups is desired. In \cite{bai4086627data}, a modified version of DBSCAN (iterative and multi-attribute) was used to cluster the different areas of the port, with the aim of improving organization and resolving port congestion. Algorithms such as GaussianMixture are used to perform segment analysis, where individuals are defined by their movement routines, and the data is related to the frequency and period of stay in different areas. From the movement information provided by smart cards, several papers apply this algorithm to identify market segments based on temporal travel patterns \cite{cats2022unravelling}, define tourist patterns based on frequency and areas where transactions are made \cite{gutierrez2020profiling} or identify changes in functional areas of cities over time \cite{wang2021identifying}.

Few works related to clustering analysis in mobility use LPR cameras as the main source of information \cite{yao2022analysis}. For example, \cite{yao2022analysis} analyze the commuting patterns constructing the spatio-temporal similarity matrix using the dynamic time warping (DTW) algorithm; and afterward, analyze the characteristics of commuting patterns with the density-based spatial clustering of applications with noise (DBSCAN) algorithm. Similarly, \cite{yao2022understanding} analyzes the change in traffic patterns during the pandemic using K-Means. However, none of these works combine LPR data with vehicle provenance nor study the touristic behavior of the vehicle. Likewise, none of them compares the suitability of using different clustering algorithms.

\section{Traffic management and business strategy in villages}

Traffic management and business strategy in rural villages is a crucial topic for the development of rural policies in Europe. The implementation of smart villages presents distinct challenges in both central and peripheral rural areas at the European and national levels. It is necessary to have an integrated vision that takes into account the specific issues, needs and expectations of each country. In this context, Spain finds itself in a situation of dual rural periphery: European and national. There are two paths towards smart villages: the horizontal connection between territories and the businesses that promote diversification \cite{paniagua2020smart}. Depopulated areas are attractive for leisure, work, and retirement. The challenge lies in developing intelligent and competitive policies that encompass all European rural areas, as well as implementing intelligent and competitive policies and strategies in depopulated zones. These policies may include the development of sustainable indicators to avoid overtourism \cite{pechlaner2019overtourism}, while businesses can leverage data on the frequency and duration of tourist visits to determine optimal opening hours and the types of goods and services to offer \cite{xie2022marketing}. Data on the modes of transportation used by visitors can also provide insights into the necessary transportation infrastructure to support businesses in the area.

Analyzing traffic patterns can also help businesses to better understand the impact of tourism on their operations. By identifying peak tourist seasons and the types of tourists that frequent the area, businesses can adapt their strategies accordingly \cite{blazquez2021identifying}. For instance, they can adjust their marketing campaigns to target specific groups of tourists or offer promotions during off-peak seasons to attract more visitors. The highlights the importance of data-driven decision-making in developing strategies that are tailored to the needs of visitors and the local community \cite{neubig2022data}. Such analyses also provide policymakers with insights to understand mobility patterns in environmentally sensitive areas, ultimately leading to better planning and management of transportation infrastructure.

\color{black}
